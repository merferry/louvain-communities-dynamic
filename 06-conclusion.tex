This study addressed the design of a high-speed community detection algorithm in the batch dynamic setting. First, we presented our optimized parallel implementations of the \textit{Louvain} and \textit{LPA} algorithms. These implementations identified communities in $6.2$ seconds and $2.7$ seconds, respectively, on a single 64-core CPU when processing an undirected web graph with $1.9$ billion edges.

Next, we discussed our \textit{Dynamic Frontier} approach (\Fro{}). Given a batch update of edge deletions and insertions, this approach addresses the issue of finding and processing only an appropriate set of affected vertices with minimal overhead. We tested it using parallel implementation of the \textit{Louvain} (\FroLou{}) and \textit{LPA} (\FroLPA{}) algorithms, compared it to two other dynamic approaches, \textit{Naive-dynamic} (\Nai{}) and \textit{Dynamic $\Delta$-screening} (\Del{}) \cite{com-zarayeneh21}, and demonstrated an improved performance of up to $1.5\times$ with \textit{Louvain} (\FroLou{}) and $10.0\times$ with \textit{LPA} (\FroLPA{}), compared to the best of the other two dynamic approaches.

Finally, we presented our novel \textit{Dynamic Frontier}-based \textit{Hybrid Louvain-LPA} (\FroHyb{}) that combines \textit{Louvain} and \textit{LPA} into a hybrid method. It leverages the advantages of both algorithms and addresses their respective limitations. We show that this approach produced high-quality results while being $7.5\times$ faster than \textit{Dynamic Frontier}-based \textit{Louvain} (\FroLou{}).

Our \textit{Dynamic Frontier} approach (\Fro{}) incrementally identifies fewer affected vertices compared to \textit{Dynamic $\Delta$-screening} (\Del{}), and converges in fewer iterations than static algorithms. This translates to decreased workload, leading us to anticipate similar performance benefits across other shared memory programming models. We plan to explore similar algorithms on GPUs and other multi-processor systems.
