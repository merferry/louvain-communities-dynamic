A core idea for dynamic community detection, among most approaches, is to use the community membership of each vertex from the previous snapshot of the graph, instead of initializing each vertex into singleton communities \cite{com-aynaud10, com-chong13, com-cordeiro16, com-zarayeneh21}. Aynaud et al. \cite{com-aynaud10} simply run the Louvain algorithm after assigning the community membership of each vertex as its previous community membership. Chong et al. \cite{com-chong13} reset the community membership of vertices linked to an updated edge, in addition to the steps performed by Aynaud et al., and process all vertices with the Louvain algorithm.

Meng et al. \cite{com-meng16} present a dynamic Louvain algorithm with an objective of obtaining temporally smoothed community structures as the graphs evolve over time. To avoid dramatic changes in community structure, they use an approximate version of delta-modularity optimization. This approximate formulation relies on both graphs in the previous and the current snapshot with a user-defined ratio. Their algorithm demonstrates improvement over Dynamic Spectral Clustering (DSC) \cite{chi2009evolutionary} and Multi-Step Community Detection (MSCD) \cite{aynaud2011multi} in terms of runtime. In terms of quality of communities obtained (using modularity score), their algorithm outperforms DSC, and is on par with MSCD.

Cordeiro et al. \cite{com-cordeiro16} propose a dynamic algorithm with a similar objective, i.e., tracking communities over time. Their algorithm performs a local modularity optimization that maximizes the modularity gain function only for those communities where the editing of nodes and edges was performed by disbanding such communities to a lower-level network, keeping the rest of the network unchanged. They confirm supremacy of their algorithm over LabelRank \cite{xie2013labelrank}, LabelRankT \cite{com-xie13}, Speaker–Listener Label Propagation (SLPA) \cite{com-xie11}, and Adaptive Finding Overlapping Community Structure (AFOCS) \cite{nguyen2011overlapping} in terms of runtime --- and over LabelRank, LabelRankT and AFOCS in terms of modularity score.

\ignore{Yin et al. \cite{com-yin16} propose an incremental community detection method based on modularity optimization for node-grained streaming networks. This method takes one vertex and its connecting edges as a processing unit, and equally treats edges involved by same node.}

\ignore{Zhuang et al. \cite{com-zhuang19} propose DynaMo, a modularity-based dynamic community detection algorithm, aimed to detect communities of dynamic networks as effective as repeatedly applying static algorithms but in a more efficient way. DynaMo is an adaptive and incremental algorithm, which is designed for incrementally maximizing the modularity gain while updating the community structure of dynamic networks.}

Zarayeneh et al. \cite{com-zarayeneh21} propose the Delta screening approach for updating communities in a dynamic graph. This technique examines edge deletions and insertions to the original graph, and identifies a subset of vertices that are likely to be impacted by the change using the modularity objective. Subsequently, only the identified subsets are processed for community state updates --- using two modularity optimization based community detection algorithms, Louvain and Smart Local-Moving (SLM). Their results demonstrate performance improvement over the Static Louvain algorithm, and a dynamic baseline version of Louvain algorithm \cite{com-aynaud10}. They also compare the performance of their algorithm against two other state-of-the-art methods DynaMo \cite{com-zhuang19} and Batch \cite{com-chong13}, and observe improvement over the methods both in terms of runtime and modularity.

\ignore{However, the algorithms proposed by Aynaud et al. \cite{com-aynaud10} and Chong et al. \cite{com-chong13} are, what we call, Naive-dynamic approaches. This is because they end up processing all vertices in the graph (albeit for a fewer number of iterations). However, finding a subset of vertices that need to be processed can help minimize computation time, which is critical for dynamic graph algorithms. The dynamic algorithm put forth by Meng et al. \cite{com-meng16} provides temporal smoothing of community memberships. But, it does not improve upon the performance of Static Louvain algorithm. Thus it fails to satisfy one of the main objectives of a dynamic community detection algorithm, i.e., to outperform the static algorithm such that an updated community structure can be quickly obtained. Further, it obtains lower modularity scores than the Static Louvain algorithm. Meng et al. state this to be due to their algorithm trading modularity maximization for temporal smoothing of community memberships. The dynamic algorithm introduced by Cordeiro et al. \cite{com-cordeiro16} obtains similar modularity scores as the Static Louvain algorithm, but is generally slower even for small batch updates. The \textit{$\Delta$-screening} technique laid out by Zarayeneh et al. \cite{com-zarayeneh21} has the properties of a desirable dynamic community detection algorithm, i.e., it identifies a subset of vertices whose community labels are likely to change on deletion/insertion of a few edges. However, our observations indicate that it suffers from identifying far too many vertices as affected --- requiring a large amount of work to identify the new communities.\ignore{Their algorithm also does not address the possibility of isolated community splits in the presence of intra-community edge deletions.}}

\ignore{In addition, the dynamic algorithms mentioned above are \textit{sequential} \cite{com-aynaud10, com-chong13, com-meng16, com-cordeiro16, com-zarayeneh21}. There is thus a need for efficient parallel algorithms for community detection on dynamic graphs. Further, none of the works recommend reusing the previous \textit{total edge weight} of each vertex/community (required for local-moving phase of Louvain algorithm) as auxiliary information to the dynamic algorithm. Recomputing it from scratch is expensive and becomes a bottleneck for dynamic Louvain algorithm. We summarize the state-of-the-art in Table \ref{tab:compare}.}

\ignore{Reidy and Bader \cite{com-riedy13} present a parallel dynamic algorithm for community detection. However, their work has a few limitations. They do not consider cascading changes to community labels on an update or study the modularity of obtained communities. They do not use \textit{Louvain} or \textit{LPA} in arriving at the communities and hence a direct comparison with their approach is not feasible.}

\ignore{propose an incremental algorithm built upon DEMON, an overlapping community detection method, while Nath and Roy \cite{nath2019detecting} present InDEN, another incremental algorithm. However, their algorithms do not handle edge deletions.}
