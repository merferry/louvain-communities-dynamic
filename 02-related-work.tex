Identifying hidden communities within networks is a crucial graph analytics problem that arises in various domains such as drug discovery, disease prediction, protein annotation, topic discovery, inferring land use, and criminal identification. The main objective is to identify groups of vertices that exhibit dense internal connections but sparse connections with the rest of the graph \cite{com-gregory10}. However, this problem is NP-hard, and there is a lack of apriori knowledge on the number and size distribution of communities \cite{com-blondel08}.

To solve this issue, researches have come up with a number of heuristics for finding communities. These include label propagation \cite{com-raghavan07, com-gregory10}, random walk \cite{com-rosvall08}, diffusion \cite{com-kloster14}, spin dynamics \cite{com-reichardt06}, fitness metric optimization \cite{com-newman06, com-fortunato10}, statistical inference \cite{com-come15, com-newman16}, core clustering \cite{com-ruan15}, simulated annealing \cite{com-guimera05, com-reichardt06}, clique percolation \cite{com-derenyi05, com-gupta22}, information theory (infomap) \cite{infomap-rosvall09, com-rita20}, and  biological evolution (genetics) \cite{com-ghoshal19, com-lu20} are studied over the decades for this problem. To evaluate the success of these methods, metrics such as the modularity score \cite{com-newman06, com-blondel08}, Normalized Mutual Information index (NMI) \cite{com-jain17, com-chopade17}, and Jaccard Index \cite{com-jain17} are often employed.

The \textit{Louvain} algorithm, based on modularity optimization, employs a greedy strategy to hierarchically merge graph vertices and extract communities \cite{com-blondel08}. It has a time complexity of $O(KM)$ (where $M$ represents the number of edges in the graph, $K$ represents the total number of iterations performed across all passes), and it efficiently identifies communities with resulting high modularity. As a result, the \textit{Louvain} method is widely favored among researchers \cite{com-lancichinetti09}.

Algorithmic improvements to the original algorithm have been proposed, which include early pruning of non-promising candidates (leaf vertices) \cite{com-ryu16, com-halappanavar17, com-zhang21, com-you22}, attempting local move only on likely vertices \cite{com-ryu16, com-ozaki16, com-zhang21, com-shi21}, ordering of vertices based on node importance \cite{com-aldabobi22}, moving nodes to a random neighbor community \cite{com-traag15}, threshold scaling \cite{com-lu15, com-naim17, com-halappanavar17}, threshold cycling \cite{com-ghosh18}, subnetwork refinement \cite{com-waltman13, com-traag19}, multilevel refinement \cite{com-rotta11, com-gach14, com-shi21}, and early termination \cite{com-ghosh18},

To parallelize the algorithm on multicore CPUs, GPUs \cite{com-cheong13}, hybrid CPU-GPUs \cite{com-bhowmik19}, multi GPUs \cite{com-cheong13, com-gawande22, com-bhowmick22}, and distributed systems \cite{com-bhowmick22}, a number of strategies have been attempted. These include parallelizing the costly first iteration \cite{com-wickramaarachchi14}, performing iterations asynchronously \cite{com-que15, com-shi21}, ordering vertices via graph coloring \cite{com-halappanavar17}, using vector based hashtables \cite{com-halappanavar17}, using adaptive parallel thread assignment \cite{com-fazlali17, com-naim17, com-mohammadi20}, using sort-reduce instead of hashing \cite{com-cheong13}, using simple partitions based of vertex ids \cite{com-cheong13, com-ghosh18}, and identifying and moving ghost/doubtful vertices \cite{com-zeng15, com-que15, com-bhowmik19, com-bhowmick22}.

The \textit{Label Propagation Algorithm (LPA)} is a method used for identifying communities or groups within a network by initializing each vertex with a unique label and diffusing these labels across the graph. It is faster and more scalable than the Louvain algorithm, as it does not require repeated optimization steps and is easy to parallelize \cite{com-newman04, com-raghavan07}.

Improvements upon the LPA include using a stable (non-random) mechanism of label choosing in the case of multiple best labels \cite{com-xing14}, addressing the issue of monster communities \cite{com-berahmand18, com-sattari18}, identifying central nodes and combining communities for improved modularity \cite{com-you20}, and using frontiers with alternating push-pull to reduce the number of edges visited and improve solution quality \cite{com-liu20}. A GPU-accelerated parallel implementation of the original LPA is available that is able to deal with large-scale datasets that do not fit into GPU memory \cite{com-kozawa17}.

A growing number of research efforts have focused on detecting communities in dynamic networks. The simplest approach is to use the community membership of vertices from the previous snapshot of the graph \cite{com-aynaud10, com-chong13, com-shang14, com-zhuang19} (which we call \textit{Naive-dynamic}). Alternatively, more advanced techniques have been employed to minimize computation by identifying a smaller subset of the graph that is affected by changes, such as moving only changed vertices \cite{com-aktunc15, com-yin16}, recomputing vertices close to an updated edge (below a given threshold distance) \cite{com-held16}, disbanding affected communities to lower-level network \cite{com-cordeiro16}, or using a dynamic modularity metric to compute community membership of vertices from scratch \cite{com-meng16}. \textit{Delta-Screening} (or \textit{$\Delta$-screening}) is a recently proposed technique that finds a subset of vertices impacted by changes in a graph using delta-modularity \cite{com-zarayeneh21}.

Significant research effort has also been dedicated to the development of dynamic label-propagation methods, due to their simplicity, efficiency, and scalability. In addition to the \textit{Naive-dynamic} approach, a number of advanced techniques been proposed. These include using the MapReduce model to efficiently adjust the communities of certain vertices based on previous intervals \cite{com-li17}, and using a stabilized label propagation process based on the static LabelRank algorithm \cite{com-xie13}. Adaptive Label Propagation Algorithm (ALPA) is another dynamic approach, which first performs a warm-up LPA on a subset of the network determined by edge deletions and insertions, followed by Local Label Propagation (LLP) which expands as a frontier of nodes that change labels and removes nodes that do not change labels \cite{com-han17}.





\paragraph{Louvain algorithm \cite{com-blondel08}}

The Louvain algorithm is a popular algorithm to efficiently identify communities with a high modularity. As a result, it is widely favored among researchers \cite{com-lancichinetti09}. Algorithmic improvements to the \Lou{} method include early pruning of non-promising candidates \cite{com-halappanavar17}, threshold scaling \cite{com-naim17, com-halappanavar17}, threshold cycling \cite{com-ghosh18}, and early termination \cite{com-ghosh18}.

A variety  of techniques have been studied to parallelize the \Lou{} algorithm on multicore CPUs, GPUs \cite{com-cheong13}, hybrid CPU-GPUs, multi GPUs \cite{com-cheong13, com-bhowmick22}, and distributed systems \cite{com-bhowmick22}, These include parallelizing the costly first iteration \cite{com-wickramaarachchi14}, ordering vertices via graph coloring \cite{com-halappanavar17}, using vector-based hashtables for data caching \cite{com-halappanavar17}, using adaptive parallel thread assignment \cite{com-naim17}, using sort-reduce instead of hashing \cite{com-cheong13}, using simple partitions based on vertex ids \cite{com-cheong13, com-ghosh18}, and identifying and moving ghost/doubtful vertices \cite{com-bhowmick22}.

A growing number of research efforts have focused on detecting communities in \textit{dynamic networks}. A core idea among most approaches is to use the community membership of each vertex from the previous snapshot of the graph, instead of initializing each vertex into singleton communities \cite{com-aynaud10, com-chong13, com-cordeiro16, com-zarayeneh21}. Aynaud et al. \cite{com-aynaud10} simply run the Louvain algorithm after assigning the community membership of each vertex as its previous community membership. Chong et al. \cite{com-chong13} reset the community membership of vertices linked to an updated edge, in addition to the steps performed by Aynaud et al., and process all vertices with the Louvain algorithm.

\kk{the next two paras can be abridged -- may be just mention these two works?}
Meng et al. \cite{com-meng16} present a dynamic Louvain algorithm with an objective of obtaining temporally smoothed community structures as the graphs evolve over time. To avoid dramatic changes in community structure, they use an approximate version of delta-modularity optimization. This approximate formulation relies on both graphs in the previous and the current snapshot with a user-defined ratio. Their algorithm demonstrates improvement over Dynamic Spectral Clustering (DSC) and Multi-Step Community Detection (MSCD) in terms of runtime. In terms of quality of communities obtained (using modularity score), their algorithm outperforms DSC, and is on par with MSCD.

Cordeiro et al. \cite{com-cordeiro16} propose a dynamic algorithm with a similar objective, i.e., tracking communities over time. Their algorithm performs a local modularity optimization that maximizes the modularity gain function only for those communities where the editing of nodes and edges was performed by disbanding such communities to a lower-level network, keeping the rest of the network unchanged. They confirm supremacy of their algorithm over LabelRank, LabelRankT, Speaker–Listener Label Propagation (SLPA), and Adaptive Finding Overlapping Community Structure (AFOCS) in terms of runtime --- and over LabelRank, LabelRankT and AFOCS in terms of modularity score.

Zarayeneh et al. \cite{com-zarayeneh21} put forward a technique called Delta-screening for updating communities in a dynamic graph. This technique examines edge deletions and insertions to the original graph, and identifies a subset of vertices that are likely to be impacted by the change using the modularity objective. Subsequently, only the identified subsets are processed for community state updates --- using two modularity optimization based community detection algorithms, Louvain and Smart Local-Moving (SLM). Their results demonstrate performance improvement over the Static Louvain algorithm, and a dynamic baseline version of Louvain algorithm \cite{com-aynaud10}. They also compare the performance of their algorithm against two other state-of-the-art methods DynaMo and Batch, and observe improvement over the methods both in terms of runtime and modularity.




\paragraph{Label Propagation Algorithm (LPA) \cite{com-raghavan07}}

While \Lou{} obtains high-quality communities, we find it to be $2.3 - 14\times$ slower than \LPA{} (which obtains communities of lower quality by $3.0 - 30\%$). LPA is faster than the Louvain algorithm, as it does not require repeated optimization steps and is easier to parallelize. Improvements upon the \LPA{} include using stable max-label choosing \cite{com-xing14}, identifying central nodes and combining communities \cite{com-you20}, and using frontiers with alternating push-pull to reduce edges visited and improve solution quality \cite{com-liu20}.

For \LPA{}, a stabilized process based on LabelRank algorithm has been proposed \cite{com-xie13}. Adaptive Label Propagation Algorithm (ALPA) is another dynamic approach, which first performs a warm-up \LPA{} on a subset of the network, followed by Local Label Propagation (LLP) which expands as a frontier of nodes that change labels \cite{com-han17}. \su{To include detailed citations on LPA or other similar algorithms.}



\ignore { 
\subsection{Critical review on Louvain algorithm}

Existing works on \textit{Static Louvain} algorithm, mentioned above, propose a number of algorithmic optimizations. In addition, they propose a number of techniques for optimizing parallel implementations. However, they do not address optimization for the \textit{aggregation phase} of the Louvain algorithm --- which increasingly becomes the bottleneck after the local-moving phase of the algorithm has been optimized. Further, these techniques are scattered over a number of papers, making it difficult for a casual reader to get a grip over them.

Let us now move on to dynamic Louvain algorithms. The ones proposed by Aynaud et al. \cite{com-aynaud10} and Chong et al. \cite{com-chong13} are essentially Naive-dynamic algorithms, as they end up processing all vertices in the graph (albeit for a fewer number of iterations). However, finding a subset of vertices that need to be processed can help minimize computation time, which is critical for dynamic graph algorithms.

The dynamic algorithm put forth by Meng et al. \cite{com-meng16} provides temporal smoothing of community memberships. But, it does not improve upon the performance of Static Louvain algorithm. Thus it fails to satisfy one of the main objectives of a dynamic community detection algorithm, i.e., to outperform the static algorithm such that an updated community structure can be quickly obtained. Further, it obtains lower modularity scores than the Static Louvain algorithm. Meng et al. state this to be due to their algorithm trading modularity maximization for temporal smoothing of community memberships. The dynamic algorithm introduced by Cordeiro et al. \cite{com-cordeiro16} obtains similar modularity scores as the Static Louvain algorithm, but is generally slower even for small batch updates.

The \textit{$\Delta$-screening} technique laid out by Zarayeneh et al. \cite{com-zarayeneh21} has the properties of a desirable dynamic community detection algorithm. Despite this, our observations indicate that it suffers from identifying far too many vertices as affected --- requiring a large amount of work to identify the new communities. Their algorithm also does not address the possibility of isolated community splits in the presence of intra-community edge deletions.

In addition, the dynamic algorithms mentioned above are \textit{sequential} \cite{com-aynaud10, com-chong13, com-meng16, com-cordeiro16, com-zarayeneh21}. Thus, there is a need for efficient parallel algorithms for community detection on dynamic graphs.\ignore{We summarize the state-of-the-art in Table \ref{tab:comparision}.} Further, none of the works recommend reusing the previous \textit{total edge weight} of each vertex (which is required for local-moving phase of the Louvain algorithm) as auxiliary information to the dynamic algorithm. Recomputing it from scratch is expensive and becomes a bottleneck for dynamic Louvain algorithm.




\subsection{Critical review on LPA}

\ignore{Reidy and Bader \cite{com-riedy13} present a parallel dynamic algorithm for community detection. They do not use \textit{Louvain} or \textit{LPA} in arriving at the communities and hence a direct comparison with their approach is not feasible.}
\ignore{Reidy and Bader \cite{com-riedy13} have proposed a parallel batch dynamic algorithm for community detection. However, their work has a few limitations. They do not consider cascading changes to community labels on an update or study the quality of obtained communities.}
}
